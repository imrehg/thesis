\setcounter{chapter}{7} %this gives Chapter 8
\chapter{Conclusion}
\label{chapter:conclusion}

This thesis presents a study of various issues in the extension of ion trap quantum computing experiments to small traps and arrays of traps. Two specific trap arrays were studied in some detail by numerical modelling and one of these was developed and operated experimentally.

The first is an ion trap designed for fast ion separation (``Liverpool trap''). It is a mesoscopic ion trap, which is considered a transition stage to combine easier manufacturing with the ability to test a geometry developed for microscopic segmented designs. Chapter~\ref{chapter:simulation} presents an overview of computer modelling of ion traps, and applies the methods to analyse the behaviour of the Liverpool trap. It was found that trapping outside the trap centre will result in large, uncompensatable micromotion along the trap axis. Differences between the original design and the actual trap geometry after assembly were evaluated, with the conclusion that in the radial direction also, micromotion compensation could be difficult due to the imperfections of electrode alignment. 

The second design (``Sandia trap'') is a true microfabricated segmented ion trap. A major part of this thesis is the experimental evaluation of this trap. Chapter~\ref{chapter:apparatus} introduces the experimental equipment and several issues which arose while building up the system.

In Chapter~\ref{chapter:firstobs} the preparation of the vacuum system is detailed together with the first testing of the calcium oven and laser system to observe neutral atom fluorescence. The neutral atom density was inferred as a function of calcium oven drive current. This allowed us to use a low drive current in order to limit both coating of the trap with calcium and the depletion of the oven. It also gave us good knowledge of the flux in the atomic beam, which is a useful diagnostic.

Previously only the NIST Ion Storage Group in the USA had been able to trap in the Sandia trap design, while the University of Innsbruck Ion Trapping Group in Austria and the University of Michigan Trapped Ion Quantum Computing Group were unsuccessful. We also experienced difficulty, though with eventual success. Chapter~\ref{chapter:waveforms} presents a numerical study of ion loading as a function of RF voltage amplitude in the Sandia trap. The calculations showed that voltages higher than 50\V\, have to be present on the RF rails to have significant trapping probability, and that high energy and low energy ions tend to be trapped with different ionization positions in the trapping region: low energy ions in the central part of the trapping region, high energy ions in the outer areas. Chapter~\ref{chapter:sandia} is concerned with the first experiments with ions in the Sandia trap, and details the successful loading of the trap. Once ion loading was reliable, micromotion compensation was performed in both radial directions and monitored using an RF-photon correlation method. A number of factors were identified as influencing the micromotion compensation. The running of the calcium oven significantly changed the compensation, while previously suspected UV charging of the electrodes due to the photoionisation laser beams was not observed. The room lighting was shown to cause an instantaneous, reversible, significant compensation change, the mechanism of which is not yet clear.

 The ion lifetime was measured with a range of parameter settings and a lifetime shortening effect of the calcium oven was established. Low oven currents significantly increased the expected lifetime. Ultimately an ion lifetime of over two hours was achieved. The axial and radial trap frequencies were measured by resonant excitation of the ion, which allowed comparison of experimental results with the predictions of the computer models of the trap. They agreed to about $10\%$. Loading of multiple ions in a reliable fashion was attempted. Initially it was impossible to load pairs of ions. When the axial vibrational frequency of the trap was lowered from 800-900\kHz\, to 300\kHz, ion pair loading was successful, though their lifetime was then only of the order of a minute.

In Chapter~\ref{chapter:heating} the characterization of the Doppler-cooling beams is presented, followed by details of two additional experiments not previously attempted using the Sandia trap. The first was to measure the motional heating rate of the ion by time-resolved recording of fluorescence during Doppler-cooling, after the ion was allowed to heat up by turning off the Doppler-cooling beams. The analysis brought to light a high heating rate due to high frequency noise on the DC electrodes. Installing low-pass filters on the DC voltage lines reduced the heating rate by a factor of 3, to $60\pm8.5~\K/\s$. Additional experiments showed that this value is fairly stable day-to-day. Ions were reliably stored up to about 755\ms\, with the cooling beams extinguished. To compare the Sandia trap with other trap designs, the electric field noise spectral density was calculated. It is shown that, while it has a very high heating rate, the Sandia trap fits into the general trend observed in ion traps, where the heating rate scales as $\omega^{-2.4}$  with the trap frequency $\omega$, and as $d^{-4}$ with the distance $d$ of the ion and the nearest electrode. Our trap had an axial vibrational frequency $\omega/2\pi$ = 300-900\kHz\, and characteristic distance of $d = 100\um$, so a high heating rate is not surprising.

Chapter~\ref{chapter:waveforms} contains a description of a precise ion shuttling method. The idea is to engineer the time dependence of the moving potential well to transport the ion with velocity corresponding to a predefined time-dependent function. This allows motional heating to be avoided, up to the limit set by the achievable precision of the voltage waveforms and modelling of the trap.  Chapter~\ref{chapter:heating} contains results of initial ion shuttling experiments in our lab. A single ion was transported to a distance of 360\um\, (the maximum possible in the Sandia trap) along the trap axis and back, repeatedly, on a few millisecond time scale. When a delay was added between the outward and inward shuttle, a maximum delay time of 120\ms\, was achieved, over which the shuttle was still 100\% reliable over 500+ repeats. Longer delay times resulted in ion loss. Comparing the 755\ms\ longest storage time in the absence of cooling with the maximum shuttle delay of 120\ms, it is clear that the shuttling results in a less stable system. This could be either owing to the method of shuttle, or the higher electric field noise at the position away from the trap centre, or a combination of the two.

Overall, the results suggest that microfabricated ion trap arrays are feasible and should be pursued, and that the particular designs considered here merit further studies.

\section{Further plans}
\label{sec:plans}

Several of the experiments described in the thesis are to be further developed. The Liverpool trap was recently made operational, successfully trapping ions, and initial testing is under way. It will allow testing of ion shuttling and ion-pair separation more readily than the Sandia trap, due to its larger trap depth and lower noise. It has already been shown that loading multiple ions is much easier and they have a long lifetime (several hours) and lower heating rate, even without precise micromotion compensation. Faster electronics are to be installed to conduct fast ion separation experiments, implementing the method presented in Section~\ref{sec:preciseshuttle}.

Despite its apparent limitations, our work has shown that the Sandia microfabrication method produced a viable trap. It is not as stable as it needs to be, e.g. having high heating rate for single ions and short lifetime for multiple ions, but these issues will partly be addressed in the new version of the trap, such as through better coating of surfaces which promises to reduce stray charge build-up and heating. Further studies are warranted to measure the heating rate of this and other new traps and implement cooling to the motional ground state. 

The new experimental setup developed for this work has the advantage that it allows us to change traps relatively easily, e.g. currently the Sandia trap is switched out in order to test a different ion trap from Lucent Technologies. Other trap designs with 2D structure, such as Y-junctions, are likely to be tested in the near future. Extending the experiments such as entanglement and quantum gate operations from single trapping region to trap arrays is also in the planning stage.

In the medium and longer term, further elements of the ion trap quantum computing paradigm can be realized.

