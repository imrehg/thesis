\chapter*{Abstract}

Implementing Segmented Ion Trap Designs For Quantum Computing. \\
A thesis submitted for the degree of Doctor of Philosophy \\
Trinity Term 2008 \\
\\
Gergely Imreh \\
St. Peter's College, Oxford \\
\\

With all the key elements of quantum computing in ion traps demonstrated by the research community, the focus is now placed on building more sophisticated traps with larger numbers of ions to allow practical scale information processing. One promising avenue is to store ions in and shuttle them between many independent traps which serve as potential interaction sites.

The core of the work described in this thesis is the experimental evaluation of a microfabricated segmented ion trap, built by Sandia National Laboratories (``Sandia trap''). These experiments required construction of a wholly new optical setup including laser and detection systems, a vacuum system and control electronics. Among our experimental achievements were: successful loading of single and pairs of ions in the microscale trap, measurement of ion storage lifetime, measurement of the motional heating rate with a time-resolved Doppler-cooling method --- which showed above than average heating, and implemented a single-ion shuttling method --- which reliably transferred the ion through a distance of 360\um\, (two DC electrode widths away) and back. These results have been used to improve the next version of the Sandia trap design.

We also used computer modelling to study several aspects of ion traps: a mesoscopic ion trap designed for fast ion separation, simulated ion loading to quantify requirements for successful trapping in small and shallow traps, and analyzed a precise shuttling method --- where the time dependence of the trapping potential is engineered such that there is minimal motional heating.

%
%The core of the work described in this thesis is the experimental evaluation of a microfabricated segmented ion trap, built by Sandia National Laboratories (``Sandia trap''). We built a new optical setup (including lasers and detection system), vacuum system and control electronics for these experiments. We achieved successful ion loading and studied the ion storage lifetime. We implemented time-resolved Doppler-cooling to measure the motional heating rate, which showed that the current version of the Sandia trap suffers from high heating. A single-ion shuttling method was tested, which reliably transferred the ion through a distance of 360\um\, (two DC electrode widths away) and back. Preliminary experiments to load pairs of ions were conducted, with success but short lifetime. These results have been used to improve the next version of the Sandia trap design.
%
%Several theoretical studies are also presented. Computer simulations of a different, mesoscopic ion trap, designed for fast ion separation, were conducted. The simulation identified a number of limitations of the design, such as high axial micromotion outside the trap centre. Ion loading was modelled to quantify requirements for successful trapping in small and shallow traps. A precise shuttling method is explored, where the time dependence of the trapping potential is engineered such that there is no motional heating, except that caused by noise and imprecision in the experimental implementation.

The results show that ion trap arrays at the 100\um\, distance scale are feasible and suggests that such multiple trap designs merit further study.







%
%With all the key elements of quantum computing in ion traps having been successfully demonstrated, the focus of research community is now shifting towards scaling up ion traps to allow larger scale information processing, with larger numbers of ions. One promising avenue is uses ion trap arrays to store ions in and shuttle them between many independent traps which serve as potential interaction sites.
%
%The core of the work described in this thesis is the experimental evaluation of a microfabricated segmented ion trap, built by Sandia National Laboratories (``Sandia trap''). These experiments required construction of a wholly new optical setup including lasers systems and detection system, vacuum system and control electronics. Among our experimental achievements were: successful ion loading in the microscale trap, measurement of ion storage lifetime, measurement of motional heating rate with a time-resolved Doppler-cooling method, which showed above than average heating, implemented a single-ion shuttling method, which reliably transferred the ion through a distance of 360\um\, (two DC electrode widths away) and back, loaded pairs of ions were conducted. These results have been used to improve the next version of the Sandia trap design.
%
%We also used computer modelling to study the behaviour of ion traps: evaluated a mesoscopic ion trap designed for fast ion separation, where we identified a number of limitations, such as high axial micro-motion outside the trap centre, simulated ion loading to quantify requirements for successful trapping in small and shallow traps, and analyzed a precise shuttling method, where the time dependence of the trapping potential is engineered such that there is no motional heating, except that caused by noise and imprecision in the experimental implementation.
%
%%
%%The core of the work described in this thesis is the experimental evaluation of a microfabricated segmented ion trap, built by Sandia National Laboratories (``Sandia trap''). We built a new optical setup (including lasers and detection system), vacuum system and control electronics for these experiments. We achieved successful ion loading and studied the ion storage lifetime. We implemented time-resolved Doppler-cooling to measure the motional heating rate, which showed that the current version of the Sandia trap suffers from high heating. A single-ion shuttling method was tested, which reliably transferred the ion through a distance of 360\um\, (two DC electrode widths away) and back. Preliminary experiments to load pairs of ions were conducted, with success but short lifetime. These results have been used to improve the next version of the Sandia trap design.
%%
%%Several theoretical studies are also presented. Computer simulations of a different, mesoscopic ion trap, designed for fast ion separation, were conducted. The simulation identified a number of limitations of the design, such as high axial micromotion outside the trap centre. Ion loading was modelled to quantify requirements for successful trapping in small and shallow traps. A precise shuttling method is explored, where the time dependence of the trapping potential is engineered such that there is no motional heating, except that caused by noise and imprecision in the experimental implementation.
%
%The results show that ion trap arrays at the 100\um\, distance scale are feasible and suggests that such multiple trap designs merit further study.