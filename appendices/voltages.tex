\chapter{Calculation of voltage configurations}
\label{app:voltagecalc}

The following programs were developed to calculate voltages in the modelled ion traps. 

\section{Function to solve voltage configuration}
\label{sec:voltprepare}
The following Matlab program calculates voltages on electrodes given a number of constraints: the location and the force experienced by the ion, and that specified electrodes be kept at a fixed voltage. The potential function generated by a unit voltage on a given electrode is passed as an input parameter to this program. The output is a set of voltages satisfying the constraints. In this setting one can calculate the voltages for trapping at an arbitrary position --- as e.g. in Sections~\ref{sec:liverpoolgeom} and \ref{sec:sandiaintro}, or the voltages needed for ion shuttling --- as in Section~\ref{sec:preciseshuttle}. 

The format in which the potential functions were stored was as follows. A chosen electrode is set to 1 volt with the other electrodes grounded, and CPO was used to calculate the potential at 1000 positions along the z axis. This set of points was then fitted with an order 34 polynomial. The 35 coefficients were stored. Other methods to store the same information are possible. For example, one could pick a set of representative points along the z axis and fit a lower order polynomial for each, or one could use another curve fitting method such as a Pad\'e approximation \cite{Baker1996}.
 

\begin{mylisting}
\begin{verbatim}
function vout = voltprepare(elecs,params,backg,pots,f,x0)
% vout = voltprepare(elecs,params,backg,pots,f,x)
% 
% Input:
% ------
% elecs : electrode #s, for which the voltages have to be calculated
% params : [w0 m a0 q] to describe the system and the trapping strength one 
%        wants to achieve
% backg : [num1 volt1; num2 volt2;...] electrode numbers and voltage values for
%        those electrodes which have a constant set value
% pots : the database for the polinomials describing the potentials
% f  :  force on the ion
% x0  :  position of ion
% 
% Output:
% -------
% vout : list of voltages for electrodes in 'elecs', to fullfill the set
%       requirements
%     


%Getting parameters for system
w0 = params(1);
m = params(2);
a0 = params(3);
q = params(4);

%number of electrodes changed and static background electrodes
n = length(elecs);
bn = size(backg);
bl = bn(1);

%Distance conversion from dimensionless
x = x0*2*a0;

%Setting up the coefficient matrix for the equations
%Dimensionless form
A = zeros(n);
for i = 1:n
    for j = 1:n
        enum = elecs(j);
        v = pots(enum,:);
        for k = 1:i
            v = derivate(v);
        end
        A(i,j) = polyval(v,x);       
    end
    A(i,:) = A(i,:)*(q/(2*a0*m*w0^2))*(2*a0)^(i-1);
end

%Setting up the other part of the equation system
%Background values
b = zeros(n,1);
if bl > 0
    for i = 1:n
        
          for j = 1:bl
            enum = backg(j,1);
            v = pots(enum,:);
            for k = 1:i
                v = derivate(v);
            end
            b(i) = b(i)-backg(j,2)*polyval(v,x);
          end
          b(i) = b(i)*(q/(2*a0*m*w0^2))*(2*a0)^(i-1);
    end
end
%Required values
b(1) = b(1) + f;
b(2) = b(2) - 1;

%Solving equation system and returning the solution
al = 0.0002;  % small constant to avoid singularity

% output
vout = inv(A.' * A  + al^2*eye(n)) * A.' * b;


%% Differentiation of a polinomial
function coeff_derivative=derivate(coeff_function)
der_order=size((coeff_function),2)-1;
coeff_derivative=0;
for index=1:size((coeff_function),2)-1
    coeff_derivative(index)=der_order*coeff_function(index);
    der_order=der_order-1;
end

\end{verbatim}
\end{mylisting}



\section{Voltage to trap at arbitrary position along the trap axis}

The following program uses the routine presented in the previous section to calculate the required voltages for a single potential well with 1\MHz\, trapping frequency along the Z axis in the Liverpool trap. 

\begin{mylisting}
\begin{verbatim}
%% Voltage on all electrodes for single trapping point along the Z axis

%%%%parapmeters, 40Ca+
w0 = 1e6*2*pi;              % trap frequency
mamu = 1.67e-27;            % atomic mass unit
ml = 40*mamu;               % Ca-40
hbar = 1.05457148e-34;      % h-bar
a0 = sqrt(hbar/(2*ml*w0))   % distance scaling
q = -1.60217653e-19;        % unit charge

%Database polinomial coefficients of electrode potentials
pots = load('v_base.dat');

%Dimensionless unit of electrode separation of
xmax = 0.7525*1e-3/2/a0;

%Simulated points: between electrodes -2 to 2
x = linspace(-2.1*xmax,2.1*xmax,200);


%The background electrodes' voltage
backv = 0;


% Calculation of voltages
for i = 1:length(x)
    v = voltprepare([1 2 3 4 5 6 7],[w0 ml a0 q],[],pots,0,x(i));
    vl(i,1:7) = v(1:7);
end


%Plotting results
mk = {'b-','r--','g-.','kx-','b--','r-.','g+-'};
figure(2)
for k = 1:7
    plot(x*2*a0/0.7525*1e3,vl(:,k),mk{k},'LineWidth',2)
    hold on;
end
hold off;
 set(gca,'XTick',-2:1:2,'XTickLabel','#2|#3|#4|#5|#6','FontSize',14)
legend('v1','v2','v3','v4','v5','v6','v7')
xlim([-2.1 2.1])
xlabel('well-position (compared to electrode positions)','FontSize',16)
ylabel('Voltage on electrode (V)','FontSize',16);
\end{verbatim}
\end{mylisting}

\section{Voltages for short shuttle}

The following program calculates the voltages for preparing a 1\MHz\, potential well at a range of positions between the centre electrode and the one next to that in the Liverpool trap. Voltages changed only on those two electrodes, all other electrodes are kept constant at 12\V. 

\begin{mylisting}
\begin{verbatim}
%% Short shuttle with two electrodes 

%%%%parapmeters, 40Ca+
w0 = 1e6*2*pi;              % trap frequency
mamu = 1.67e-27;            % atomic mass unit
ml = 40*mamu;               % Ca-40
hbar = 1.05457148e-34;      % h-bar
a0 = sqrt(hbar/(2*ml*w0))   % distance scaling
q = -1.60217653e-19;        % unit charge

%Database polinomial coefficients of electrode potentials
pots = load('v_base.dat');

%Dimensionless unit of electrode separation of
xmax = 0.7525*1e-3/2/a0;

%Simulated points: between electrodes -2 to 2
x = linspace(0,xmax,200);

%The background electrodes' voltage
backv = 12;

%Calculation
for i = 1:length(x)
    v = voltprepare([4 5],[w0 ml a0 q],[1 backv; 2 backv; 3 backv; 6 backv; 7 backv],pots,0,x(i));
    vl(i,1:2) = v(1:2);
end

%Plotting
mk = {'kx-','b--','r-.'};
figure(2)
for k = 1:2
    plot(x*2*a0*1e3,vl(:,k),mk{k},'LineWidth',2)
    hold on;
end
hold off;
legend('v4','v5')
xlabel('well-position (compared to electrode positions)','FontSize',14)
ylabel('Voltage on electrode (V)','FontSize',14);
\end{verbatim}
\end{mylisting}

